\begin{table}[H]
\centering
\scriptsize
\setlength{\tabcolsep}{4pt}
\renewcommand{\arraystretch}{1.2}
\begin{tabularx}{\textwidth}{|X|X|}
\hline
\textbf{Fortalezas} & \textbf{Oportunidades} \\
\hline
\begin{minipage}[t]{\linewidth}
\vspace{2pt}
\begin{itemize}
    \setlength\itemsep{0pt}
    \setlength\parskip{0pt}
    \setlength\parsep{0pt}
    \item La VBC es ampliamente utilizada en plataformas en la nube como AWS, Azure y Google Cloud.
    \item Herramientas como Kubernetes y Docker se actualizan constantemente, agregando nuevas funcionalidades y mejoras.
    \item Numerosos foros, grupos y organizaciones están dedicados a mejorar y desarrollar soluciones de contenerización.
    \item Los contenedores pueden ejecutarse en múltiples entornos (local, nube, híbrido).
    \item Permiten una mejor utilización de recursos en comparación con las máquinas virtuales tradicionales.
\end{itemize}
\vspace{2pt}
\end{minipage}
&
\begin{minipage}[t]{\linewidth}
\vspace{2pt}
\begin{itemize}
    \setlength\itemsep{0pt}
    \setlength\parskip{0pt}
    \setlength\parsep{0pt}
    \item Los contenedores son ligeros y comparten el kernel del sistema operativo, reduciendo el consumo de recursos.
    \item Una aplicación en contenedor puede ejecutarse en cualquier entorno compatible sin modificaciones.
    \item Permiten implementar y gestionar aplicaciones de manera escalable y eficiente.
    \item Los contenedores permiten que las aplicaciones se ejecuten de manera independiente, evitando conflictos entre ellas.
    \item Herramientas como Docker Compose y Kubernetes facilitan la gestión automatizada de contenedores.
\end{itemize}
\vspace{2pt}
\end{minipage}
\\
\hline
\textbf{Debilidades} & \textbf{Amenazas} \\
\hline
\begin{minipage}[t]{\linewidth}
\vspace{2pt}
\begin{itemize}
    \setlength\itemsep{0pt}
    \setlength\parskip{0pt}
    \setlength\parsep{0pt}
    \item Los contenedores comparten el mismo núcleo del sistema operativo host, lo que podría permitir que una vulnerabilidad afecte a todos los contenedores.
    \item A medida que crece la infraestructura, gestionar múltiples contenedores y sus redes se vuelve complejo.
    \item A diferencia de las máquinas virtuales tradicionales, los contenedores son efímeros por diseño, lo que complica el manejo de datos persistentes.
    \item No todos los sistemas y aplicaciones son compatibles de manera nativa con contenedores, lo que requiere configuraciones específicas.
    \item La seguridad y rendimiento de los contenedores están directamente relacionados con el sistema operativo subyacente.
\end{itemize}
\vspace{2pt}
\end{minipage}
&
\begin{minipage}[t]{\linewidth}
\vspace{2pt}
\begin{itemize}
    \setlength\itemsep{0pt}
    \setlength\parskip{0pt}
    \setlength\parsep{0pt}
    \item Si no se aplican buenas prácticas, los contenedores pueden ser vulnerables a ataques.
    \item El uso de soluciones propietarias como AWS o Azure puede generar dependencia tecnológica.
    \item Las tecnologías de contenerización evolucionan rápidamente, lo que puede dejar obsoletas algunas soluciones.
    \item Un mal diseño puede llevar a un consumo ineficiente de recursos.
    \item Aunque existen buenas prácticas, no hay una regulación estándar única para la gestión de contenedores.
\end{itemize}
\vspace{2pt}
\end{minipage}
\\
\hline
\end{tabularx}
\caption{Tabla de matriz DOFA para el cuadrante gartner}
\label{tab:matriz-dofa}
\end{table}