\ChapterImageStar[cap:glosario]{Glosario}{./images/fondo.png}\label{cap:glosario}
\mbox{}\\
En este apartado se encuentran términos clave y conceptos relevantes utilizados a lo largo de este proyecto.

\section*{B}
\begin{description}
  \item[Benchmarking:] Mide el rendimiento o el grado de éxito alcanzado en comparación con otras empresas para una actividad, flujo de valor u otros factores de interés determinados. Estas medidas se convierten en la base para el análisis y el rediseño \citep{PeterWootton2024}.
\end{description}

\section*{C}
\begin{description}
  \item[Cloud Computing:] La computación en la nube es un modelo que permite el acceso a la red, ubicuo, práctico y bajo demanda, a un conjunto compartido de recursos informáticos configurables que pueden aprovisionarse y liberarse rápidamente con un mínimo esfuerzo de gestión o interacción con el proveedor de servicios \citep{Mell2011}.
\end{description}

\section*{E}
\begin{description}
  \item[Escalabilidad:] El escalado automático en computación se refiere al ajuste automático de los recursos informáticos a medida que aumenta la carga de trabajo. Los servicios en la nube aumentan automáticamente sus recursos informáticos en respuesta al aumento de la carga de trabajo, las solicitudes y las actividades. Como parte de este proceso, se asignan servidores adicionales, se asignan recursos de memoria y se gestionan los requisitos de red \citep{TARI2024100650}.
\end{description}

\section*{H}
\begin{description}
  \item[Hypervisor:] Es responsable de crear, administrar y programar máquinas virtuales, que representan máquinas reales para los sistemas operativos que se ejecutan en ellas \citep{Cinque2024}.
\end{description}

\section*{P}
\begin{description}
  \item[Private Cloud:] Una nube privada virtual se refiere a una nube privada alojada en un entorno de nube pública o compartida. Permite la conexión entre la infraestructura heredada y los servicios en la nube mediante una conexión de red virtual segura \citep{Collins2016}.
  
  \item[Producto mínimo viable (PMV):] El producto mínimo viable es aquella versión de un nuevo producto que permite a un equipo recopilar la máxima cantidad de aprendizaje validado sobre los clientes con el menor esfuerzo \citep{Ries2020}.
\end{description}

\section*{V}
\begin{description}
  \item[Virtualización:] Virtualización significa máquina virtual, que no existe pero proporciona todas las facilidades del mundo real, que se utilizan para mejorar la eficiencia de la computación en la nube \citep{Meena2021}.
\end{description}