\ChapterImageStar[cap:objetivos]{Objetivos}{./images/fondo.png}\label{cap:objetivos}
\mbox{}\\
En este capítulo se establece un conjunto de objetivos que orientan el desarrollo del trabajo, articulando el propósito general con metas específicas que permiten su cumplimiento de manera sistemática. Estos objetivos se centran en la definición, análisis y validación de una arquitectura basada en tecnologías de virtualización con contenedores (\VBC), con el fin de responder a las necesidades y oportunidades del Grupo de Investigación en Redes, Información y Distribución (\GRID). 

\section{Objetivo general}
\addcontentsline{toc}{section}{Objetivo general}\label{cap:objetivoGeneral}

Especificar una arquitectura de tecnologías de virtualización basadas en contenedores (\VBC), evaluando sus características a través de un benchmarking, seleccionando la que mejor se adapte a la necesidad, problema y oportunidad del \GRID\ (Grupo de Investigación en Redes, Información y Distribución), haciendo un análisis \DAR\ e implementando un producto mínimo viable (\PMV).

\section{Objetivos específicos}
\addcontentsline{toc}{section}{Objetivos específicos}\label{cap:objetivosEspecificos}
\begin{itemize}
    \item Reconocer necesidades del \GRID\ con relación a las tecnologías de virtualización basadas en contenedores.
    \item Identificar las tecnologías de virtualización basadas en contenedores.
    \item Caracterizar tecnologías de virtualización basadas en contenedores.
    \item Seleccionar un conjunto de tecnologías de contenedores para realizar pruebas de concepto.
    \item Diseñar una especificación arquitectónica para las herramientas seleccionadas.
    \item Implementar el prototipo funcional.
    \item Validar casos con relación a la necesidad del cliente.
\end{itemize}